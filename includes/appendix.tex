\section{Appendix}

In this section, we informally discuss how we established
equisatisfiability guarantees regarding the transformations provided
in \fuzzer{}.  The transformations work over boolean, integer, string
literals and expressions, and regular expressions. To shorten our
proofs, we make the following simplifications:

\begin{enumerate}
  \item
    Since many of the operators can be expressed in terms of others,
    our proofs will only consider a subset of the operators. In
    particular we drop the following from consideration:
    \begin{itemize}
      \item \texttt{or} (boolean or)
      \item \texttt{=} for boolean expressions
      \item \texttt{<=}, \texttt{>}, \texttt{>=}
      \item \texttt{str.prefixof}, \texttt{str.suffixof}, \texttt{str.at}, \texttt{str.contains}
      \item \texttt{re.range}, \texttt{re.+}, \texttt{re.allchar}, \texttt{re.all}
    \end{itemize}

  \item
    In the above, we assume that \texttt{str.at} is expressible using
    \texttt{substring} although the semantics slightly differ when the index is
    out of bounds.
\end{enumerate}

\begin{definition}
  A \emph{model} is a mapping from boolean, integer, and string variables to
elements in the domain of the universe of the corresponding type.
\end{definition}

\begin{definition}
  A \emph{formula} $P$ satisfies a model $m$ if $P$ evaluates to
  \texttt{true} under the model $m$. The function $\eval (P)$
  evaluates a formula under a model; the specific model used is left
  implicit as it is clear from context.
\end{definition}

The proofs all follow a very similar structure. For each
transformation \textit{Trans}, we show that a formula $P$ is
satisfiable by a model $m$ if and only if $\textit{Trans}(P)$ is
satisfiable by $\textit{Trans}(m)$, where $\textit{Trans}(m)$ is the
transformation applied to all the constants in the model $m$. (The
exception here is the \textit{Multiply} operator as discussed below.)

To show the above, we show that for all expressions $e$,
$\textit{Trans}(\eval (e)) = \eval (\textit{Trans}(e))$. This is
proven using a straightforward induction on the structure of
expressions. Since the transformations fix boolean constants, we have,
$\textit{Trans}(\eval (P)) = \eval (P)$, and hence if the original
problem is satisfiable so is the transformed problem.

For transformations that have an inverse, such as \textit{Translate}
and \textit{Reverse} this also shows that if the transformed problem
is satisfiable, then the original problem is satisfiable, proving that
they are equisat transformations. \textit{Multiply} (as defined here)
does not have an inverse and hence it only guarantees to take
satisfiable problems to satisfiable problems. This can be seen in the
problem instance $y<x<y+1$ which is UNSAT, but multiplying by 2
transforms the problem into $y<x<y+2$ which is SAT.

\subsection{Translate}
We now prove that \textit{Translate} transforms constraints into
equisatisfiable constraints. From above, we only need to show that for
every expression $e$, the following is true: $\textit{Translate}(\eval
(e))$ = $\eval (\textit{Translate}(e))$. For now, we assume that
\textit{Translate} fixes the digits pointwise and we prove the
stronger fact that $\textit{Translate}$ also fixes characters
pointwise in any string expression. (i.e., in the bijections of
character sets used by $\textit{Translate}$ the digits are not mapped
to any other character.)

It is clear that for literals $l$, $\textit{Translate}(\eval (l)) =
\eval (\textit{Translate}(l))$. Then, assuming that the it holds for
smaller expressions, we show that it holds larger expressions $e$
created using the constructs \texttt{and}, \texttt{=} (for strings and
integers), \texttt{+}, \texttt{-}, \texttt{*}, \texttt{str.len},
\texttt{str.indexof}, \texttt{str.++}, and \texttt{str.substr}.

For \texttt{str.in.re} we had to show that a string literals $s$, is a
member of a regular expression $r$ if and only if
$\textit{Translate}(s)$ is a member of $\textit{Translate}(r)$ by
induction on regular expressions. For the inductive cases of
\texttt{str.to.int}, and \texttt{str.from.int} we needed to use the
fact that the digits were fixed pointwise.

The restriction that \textit{Translate} must fix the digit characters
can be relaxed if we are translating problem instances that do not use
\texttt{str.to.int} or \texttt{str.from.int}. Without these operators
we can show $\textit{Translate}(\eval (e))$ = $\eval
(\textit{Translate}(e))$ directly by induction without having to worry
about fixing digits pointwise.

\subsection{Reverse}
\textit{Reverse} does not transform constraints that use
\texttt{str.replace}, \texttt{str.indexof}, \texttt{str.to.int}, and
\texttt{str.from.int}. These operators do not have a well-defined
notion of a reverse. The required lemma is proved by induction on the
rest of the operators similar to proof for \textit{Translate}.

We end our discussion on \textit{Reverse} by noting that although it
disallowed \texttt{str.replace}, it only did so because it didn't have
a natural notion of a reverse in the theory we were working with.
%% If we expand our theory to also include an operator that replaces
%% the last instance, we can transform problems that use
%% \texttt{str.replace}.

\subsection{Multiply}
\textit{Multiply} does not transform constraints that use \texttt{*},
\texttt{str.indexof}, \texttt{str.to.int}, and
\texttt{str.from.int}. These operators do not have a well-defined
notion of a multiply. The required lemma is proved by induction on the
rest of the operators similar to the above proofs.

We disallow \texttt{*} for \textit{Multiply} because of problems
instances such as \texttt{(= 10 (* 2 5))}. A naive \textit{Multiply}
by $3$ would transform this satisfiable problem into the unsatisfiable
\texttt{(= 30 (* 6 15))}.
