\section{Introduction}

In recent years, many algorithms for solving string constraints have
been developed and implemented in SMT solvers such as
Norn~\cite{norn}, CVC4~\cite{cvc4}, and Z3 (e.g., Z3str2~\cite{z3str2}
and Z3str3~\cite{z3str3}).  To validate and benchmark these solvers,
their developers have relied on hand-crafted input
suites~\cite{cvc4-tests,z3str3-tests,z3str2-tests} or real-world
examples from a limited set of industrial
applications~\cite{kaluza,kausler}. These test suites have helped
developers identify implementation defects and develop more
sophisticated solving heuristics. Unfortunately, as more features are
added to solvers, these benchmarks often remain stagnant, leaving
increasing functionality untested.  As such, there is an acute need
for a more robust, inexpensive, and automatic way of generating
benchmarks to test the correctness and performance of SMT solvers.

Fuzzing has been used to test all kinds of software including SAT
solvers~\cite{fuzzsat}. Inspired by the utility of fuzzers, we
introduce \fuzzer{} and describe its value as an exploratory testing
tool. We demonstrate its efficacy by presenting limitations it helped
discover in leading string solvers. To the best of our knowledge,
\fuzzer{} is the only tool aimed at automatic generation of string
constraints. \fuzzer{} can be used to mutate or transform existing
benchmarks, as well as randomly generate structured instances. These
instances can be scaled with respect to a variety of parameters, e.g.,
length of string constants, depth of concatenations (concats) and
regular expressions (regexes), number of variables, number of length
constraints, and many more.

\newpage
\subsubsection{Contributions}\footnote{All source code, problem suites, and
supplementary material referenced in this paper are available at the
\fuzzer{} website \cite{website}.}
\begin{enumerate}
    \item \textbf{The \fuzzer{} tool}:
        In Sect.~\ref{sec:fuzzer}, we describe a modular fuzzer that can
        transform and generate \smtfull{} string and regex
        instances. Scaling inputs (e.g., long string constants,
        deep concatenations) are particularly useful in identifying asymptotic
        behaviors in solvers, and \fuzzer{} has many options to generate them.
        We briefly document \fuzzer{}'s components and modular architecture.
        We provide example use cases to demonstrate its utility as an
        exploratory solver testing tool.

    \item \textbf{A repository of \smtfull{} instances}:
        We present a repository of \smtfull{} string and regex instance suites
        that we generated using \fuzzer{} in Sect.~\ref{sec:suites}. This
        repository consists of two categories: one with new
        instances generated by \fuzzer{} (\texttt{generated}); and another with
        transformed instances generated from a small suite of industrial
        benchmarks (\texttt{transformed}).

    \item \textbf{Experimental Results and Analysis}:
        We compare the performance of \theSolvers{} on the
        \fuzzer{} suites \theSuites{} in Sect.~\ref{sec:data}. We
        highlight these suites because they make some solvers perform poorly,
        but not others. We analyze our
        experimental results, and pinpoint algorithmic limitations
        in \us{} that cause poor performance.
\end{enumerate}
