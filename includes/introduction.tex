\section{Introduction}

    Many SMT solvers that support the theory of strings have been developed over the past decade, and several have matured to the point of being performant and useful in practice. \theSolvers{}\cite{cvc4}\cite{z3str3} are \numSolvers{} such solvers. Moreover a standard input language for problem instances, \smt{}\cite{smt}, has been developed and widely adopted. To validate their solvers, many authors have relied on hand-crafted input suites or real-world examples from industrial applications \cite{cvc4-tests} \cite{z3str3-tests} \cite{z3str2-tests}. Indeed several suites of industrial and hand-crafted problems have been compiled \cite{kaluza} \cite{kausler} to both validate solvers and benchmark their performance. However, though these suites help debug solvers, they do not age well. As solvers continue to grow, fixed benchmarks leave more and more functionality untested.

    We introduce a modular fuzzer for string solvers, \fuzzer{}, and describe its utility as an exploratory validation tool. We show its effectiveness by presenting defects we were able to find in leading solvers.

    \subsection{Contributions}

    \begin{enumerate}
        \item \textbf{The \fuzzer{} tool}: We describe a modular fuzzer which can transform and generate \smt{} string and regex problems. We briefly document its components, modular architecture, and usage patterns to demonstrate its usefulness as an exploratory tool for validating not only solver implementations, but also algorithmic assumptions.
        \item \textbf{A suite of \smt{} problems}: We present a suite of \smt{} string and regex problems we generated with \fuzzer{}. We justify the existence of each suite by a motivating example from the development of \us{}. We reveal specific defects that this suite helped uncover and fix, as well as specific conjectures that it helped validate or disprove.
        \item \textbf{Experimental Results}: We present problem classes we found that made one solver perform poorly, but not another. We show differences in solver behaviour on the suites \cHard{} and \zHard{}.
        \item \textbf{Analysis}: We analyse our experimental results by inspecting the execution traces of the solvers on the given problems. Where possible, we explain our observations. For \cvc{} we draw superficial conclusions based solely on the execution traces, and invite its authors to join us in analysing them. For \us{} we pinpoint algorithmic shortcomings that cause the poor performance.
    \end{enumerate}
