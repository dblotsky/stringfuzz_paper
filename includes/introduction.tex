\section{Introduction}

In recent years many algorithms for solving string constraints have
been developed and implemented as part of SMT solvers such as CVC4
\cite{cvc4} and Z3 (e.g., Z3str2 \cite{z3str2} and Z3str3
\cite{z3str3}). To validate and benchmark these solvers, their
developers have relied on hand-crafted input
suites~\cite{cvc4-tests,z3str3-tests,z3str2-tests} or real-world
examples from a limited class of industrial
applications~\cite{kaluza,kausler}. These test suites have helped
developers identify implementation defects and develop more
sophisticated solving heuristics. Although extremely helpful, these
suites are of limited value since they do not test the vast majority
of interesting behaviors in solvers. There is an acute need for more
robst, inexpensive, and automatic way of generating test suites to
test SMT solvers, that can augment hand-crafted as well as application
tests.
    
Random and mutation fuzzing are widely used test all kinds of software
including SAT solvers~\cite{}. Inspired by their utility, we introduce
a fuzzer for string solvers, \fuzzer{}, and describe its value as an
exploratory validation tool. We demonstrate the efficacy \fuzzer{} by
presenting defects and limitations it enabled us to identify in
leading string solvers, which otherwise would have been difficult to
detect. To the best of our knowledge, \fuzzer{} is the first such tool
aimed at automatic generation of string constraints. \fuzzer{} allows
to test string solvers by enabling us to mutate existing classes of
benchmarks in interesting ways, as well as generate extremal inputs
(e.g., inputs with very long strings or deep concatenations or
highly-complex mix of string operations and regular
expressions). Extremal inputs are particularly useful in empirically
understanding asymptotic behavior of solvers in a way that no other
class of instances can.

\noindent{\textbf{Contributions:}} 

\begin{enumerate}
\item \textbf{The \fuzzer{} tool}: We describe in
  Section~\ref{sec:fuzzer} a modular fuzzer which can transform and
  randomly generate extremal \smtfull{} string and regex problems. We
  briefly document its components and modular architecture. We provide
  example use cases to demonstrate its utility as an exploratory tool
  for validating solvers.
  
\item \textbf{A repository of \smtfull{} problems}: We present a
  repository of \smtfull{} string and regex problem suites we
  generated using \fuzzer{} in Section~\ref{sec:problems}. We
  demonstrate the usefulness of these suites by revealing specific
  defects they helped uncover and fix in \us{}.

\item \textbf{Experimental Results}: We compare \theSolvers{} on the
  \fuzzer{} suites and selectively present the experimental result on
  \theSuites{} in Section~\ref{sec:data}. We highlighted these suites
  because they uniquely made some solvers perform poorly, but not
  others.

\item \textbf{Analysis}: We analyze our experimental results in
  Section~\ref{sec:analysis} by inspecting the execution traces of the
  solvers on the given problems. We pinpoint algorithmic limitations
  in \us{} that cause poor performance, and analyze proposed and
  implemented enhancements that address the issues which \fuzzer{}
  helped us to identify.
\end{enumerate}
