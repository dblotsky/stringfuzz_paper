\section{Problem Suites}

    Along with this paper we present several suites of ready-made problems produced by StringFuzz. These suites can be found on \problemRepo{}.

    \subsection{\fuzzer{}-Generated}

        This suite contains the following problem sets:

        \begin{itemize}
            \item Lengths
            \item Lengths-Concats
            \item Concats
            \item Concats-Extracts
            \item Overlaps
            \item Regex
            \item Many-Regexes
            \item Regex-Deep
            \item Regex-Pair
            \item Regex-Lengths
            \item Random??
        \end{itemize}

        Most of these problem sets either have a dedicated generator in \generator{}, or are variations of one. While it is likely that some of the pre-made suites will be immediately useful in analysing a solver, we also found that mass randomisation was helpful in finding some bugs, and we encourage this pattern as well. \generator{} is designed to be used in an exploratory fashion, and the following usage has helped us in our own experiments:

\begin{verbatim}
while true; do stringfuzzg --random regex | z3str3 -in; done
\end{verbatim}

    \subsection{\fuzzer{}-Transformed}

        This suite contains the following problem sets:

        \begin{itemize}
            \item Amazon??
            \item Regex??
        \end{itemize}

        Most of these problem sets were generated by using one root problem and transforming it with \transformer{} into many similar problems.

    \subsection{Unit Tests}

        This suite covers corner cases of solver behaviour. We have used the instances in this suite to find bugs in \us{}.

    \subsection{Results}

        This section describes the bugs in \us{} that were discovered (and subsequently fixed) by testing it with the \fuzzer{} problem suites.

        \begin{enumerate}
            \item length lookup using eqc root
            \item length-testing bug
        \end{enumerate}