\section{Problem Suites}

    We generated several suites of string and regex problems using \fuzzer{}, and they are available at \problemRepo{}. They are described in more detail below.

    \subsubsection{\fuzzer{}-Generated}

        Most of the problem sets in this suite either have a dedicated generator in \generator{}, or are variations of one. This suite contains the following problem sets: \\

        \todo{Fill in descriptions.}

        \begin{tabular}{|c|c|}
            \hline
            \textbf{Name}    & \textbf{Description} \\ \hline
            Lengths          & \\ \hline
            Lengths-Concats  & \\ \hline
            Concats          & \\ \hline
            Concats-Extracts & \\ \hline
            Overlaps         & \\ \hline
            Regex            & \\ \hline
            Many-Regexes     & \\ \hline
            Regex-Deep       & \\ \hline
            Regex-Pair       & \\ \hline
            Regex-Lengths    & \\ \hline
        \end{tabular}

    \subsubsection{\fuzzer{}-Transformed}

        Most of the problem sets in this suite were generated by using one root problem and transforming it with \transformer{} into many similar problems. This suite contains the following problem sets: \\

        \todo{Fill in descriptions.}

        \begin{tabular}{|c|c|}
            \hline
            \textbf{Name} & \textbf{Description} \\ \hline
            Amazon        & \\ \hline
            Regex         & \\ \hline
        \end{tabular}

\subsection{Results}

    This section describes the bugs in \us{} that were discovered (and subsequently fixed) by testing it with the \fuzzer{} problem suites.

    \todo{Describe the length lookup bug and length-testing bug, giving the before/after commits.}

    While it is likely that some of the pre-generated suites will be immediately useful in analysing a solver, we also found that randomisation was helpful in exploratory debugging, and we encourage this pattern as well. For example the following usage can feed random problems to \cvc{} until it returns an error:

    {\scriptsize\begin{verbatim}
while stringfuzzg random | cvc4 --lang smt2.5 --tlimit=5000; do sleep 0; done\end{verbatim}}

    Some simple UNIX scripting can yield other interesting results. For example the following script finds a regex problem on which \us{} times out after 5 seconds:

    {\scriptsize\begin{verbatim}while true; do
    stringfuzzg -r regex -c -a increasing -r 2 -t 5 -d 3 > problem.smt25
    result=`z3str3 -T:5 problem.smt25`
    echo $result
    if [ "$result" == "timeout" ]; then
        cat hard-problem.smt2
        break
    fi
done\end{verbatim}}
