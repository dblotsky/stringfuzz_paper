\section{\fuzzer{}}
\label{sec:fuzzer}
\fuzzer{} is implemented as a Python package, and comes with several
executables to generate, transform, and analyze \smtfull{} string
problems. It can either be installed from source, or from the Python
PIP package repository. \fuzzer{} is organized to be easily extended:
all generators and transformers are independent Python modules. Its
components are also implemented as \unix{} ``filters'', to enable easy
integration with other tools (including themselves). For example, the
outputs of generators can be piped into transformers, and transformers
can be chained to produce a stream of tuned inputs to a
solver.

\vspace{0.1cm}
\noindent{Specifically, \fuzzer{} is composed of the following tools:}

\noindent{\textbf{\generator{}:}} This tool generates \smt{}
problems. It supports several generators and options that specify its
output. Details can be found in Table~\ref{tbl:generators}.
   
\begin{table}[t]
%\vspace{-0.25in}
\centering
\caption{\generator{} generators.}
\label{tbl:generators}
{
  \footnotesize
  \bgroup
  \def\arraystretch{1}
  \resizebox{0.8\columnwidth}{!}{
    \begin{tabular}{|l|l|}
      \hline
      \textbf{Name}        & \textbf{Generated Problems have ...} \\ \hline
      \textit{concats}     & Long concats and optional random extracts. \\ \hline
      \textit{lengths}     & Many variables (and their concats) with length constraints. \\ \hline
      \textit{overlaps}    & An expression of the form A.X = X.B. \\ \hline
      \textit{equality}    & An equality among concats, each with variables or constants. \\ \hline
      \textit{regex}       & Regexes of varying complexity. \\ \hline
      \textit{random-text} & Random, likely syntactically \textit{in}valid text. \\ \hline
      \textit{random-ast}  & Random, but semantically \textit{valid} text. \\ \hline
    \end{tabular}
  }
  \egroup
%\vspace{-0.2in}
}
\end{table}

\noindent{\textbf{\transformer{}:}This tool transforms \smt{}
  problems. It supports several transformers and options that specify
  its output and input, which are explained in
  Table~\ref{tbl:transformers}. Note that transformers
  \textit{translate} and \textit{reverse} also preserve problem
  satisfiability \cite{ifaz}.

\begin{table}[t]
%\vspace{-0.25in}
\centering
\caption{\transformer{} transformers}
\label{tbl:transformers}
{
  \footnotesize
  \bgroup
  \def\arraystretch{1}
  \resizebox{0.8\columnwidth}{!}{
    \begin{tabular}{|l|l|}
      \hline
      \textbf{Name}        & \textbf{The transformer ...} \\ \hline
      \textit{fuzz}        & Randomly replaces literals and operators with similar ones. \\ \hline
      \textit{graft}       & For each sort, swaps a random non-leaf node with a leaf node. \\ \hline
      \textit{multiply}    & Multiplies each number by N, and repeats each character N times. \\ \hline
      \textit{nop}         & Does nothing. \\ \hline
      \textit{reverse}     & Reverses all string literals and concat arguments. \\ \hline
      \textit{rotate}      & Performs a random syntax tree rotation on two concats. \\ \hline
      \textit{translate}   & Maps every character X to a different character Y. \\ \hline
      \textit{unprintable} & Randomly replaces all characters in literals with unprintable ones. \\ \hline
                \end{tabular}
  }
  \egroup
  %\vspace{-0.2in}
}
\end{table}  
    
\noindent{\textbf{\texttt{stringstats}:}} This tool takes an \smt{}
problem as input and outputs its properties: the number of
variables/literals, the max/median syntactic depth of expressions, the
max/median literal length, etc.

\noindent{\textbf{Regex Fuzzing Capabilities:}} \fuzzer{} can generate
and transform problems with regular expression constraints. The
command \texttt{stringfuzzg regex} invokes the regex problem
generator. It generates a problem of the form:

\begingroup
\footnotesize
\begin{align*}
  & \texttt{(assert (str.in.re X}\; R_0\; \texttt{))} \\
  & \texttt{(assert (str.in.re X}\; R_n\; \texttt{))}* \\
  & \texttt{(assert (<= Min (str.len X)))}? \\
  & \texttt{(assert (<= (str.len X)) Max)}?
\end{align*}
\endgroup

where $R_i \in RegEx$, and $Min, Max \in Int$. More simply, the
problem is a set of one or more regex constraints on a single
variable, with optional maximum and minimum length constraints. The
regex constraints $R$ are each of the form:

\begingroup
\footnotesize
\begin{align*}
  & \texttt{(re.++}\; \ T_0\; \ T_1\; \texttt{...}\; T_n\; \texttt{)}
\end{align*}
\endgroup
and each $T_i$ is a recursive term of the form:
\begingroup
\footnotesize
\begin{align*}
  & \texttt{(re.*}\; T_{i_j}\; \texttt{) | (re.+}\; T_{i_j}\;
  \texttt{) | (re.union}\; T_{i_{j_1}}\; T_{i_{j_2}}\; \texttt{)}
\end{align*}
\endgroup

where $j$ is the specified depth of recursion. Terms at depth 0 are
regex constants. Informally, this form describes a concatenation of
regex terms. Each term is a random nested regex operator (chosen from
regex Kleene star, repetition, and union), up to a specified depth,
terminating in a regex literal. Below are several example regexes
(with spacing added for clarity) of depth 2 produced by this scheme:

\begingroup
\footnotesize
\begin{align*}
  & ((\texttt{a}|\texttt{b})|(\texttt{cc})+)\quad
  ((\texttt{ddd})*)+\quad ((\texttt{ee})+|(\texttt{fff})*)
\end{align*}
\endgroup

\fuzzer{} can also transform the instances it generates. All of the
transformers in \transformer{} can operate on regex terms. They can
multiply and permute regex literals (e.g. \texttt{reverse},
\texttt{multiply}, \texttt{translate}), randomly replace regex
operators (e.g. \texttt{fuzz}), and manipulate the syntax tree
(e.g. \texttt{graft}, \texttt{rotate}).

\vspace{0.2cm}
\noindent{\textbf{Example Use Cases:}} \fuzzer{} can be used to
generate benchmark suites in a batch mode, or can be used in an online
randomized fashion for exploratory debugging. For example, \fuzzer{}
can be invoked as follows to feed random problems to \cvc{} until it
returns an error:

\begin{center}
%\vspace{-0.05in}
\begin{varwidth}{\linewidth}
{\scriptsize\begin{verbatim}while stringfuzzg -r random | tee
  problem.smt25 | cvc4 --lang smt2.5 --tlimit=5000; do sleep 0 done
\end{verbatim}}
\end{varwidth}
\end{center}

Some \unix{} scripting can yield other interesting results. For
example, the following script finds a regex problem on which \us{}
times out after 5 seconds:

\begin{center}
%\vspace{-0.1in}
\begin{varwidth}{\linewidth}
        {\scriptsize\begin{verbatim}while true; do
    stringfuzzg -r regex -c -a increasing -r 2 -t 5 -d 3 > problem.smt25
    result=`z3str3 -T:5 problem.smt25`
    echo $result
    if [ "$result" == "timeout" ]; then
        cat problem.smt25
        break
    fi
done\end{verbatim}}
\end{varwidth}
\end{center}

% \texttt{stringstats} is handy when analyzing a problem.    
%     \begin{description}

%         \item[\generator{}] \hfill \\
%             This tool generates \smt{} problems. It supports several generators and options that specify its output. Table~\ref{tbl:generators} describes them.
%             \hfill \\
% 
%             \begin{table}
%                 \centering
%                 \begin{tabular}{|l|l|}
%                     \hline
%                     \textbf{Name}        & \textbf{Generated Problems have ...} \\ \hline
%                     \textit{concats}     & Long concats and optional random extracts. \\ \hline
%                     \textit{lengths}     & Many variables (and their concats) with length constraints. \\ \hline
%                     \textit{overlaps}    & An expression of the form A.X = X.B. \\ \hline
%                     \textit{equality}    & An equality among concats, each with variables or constants. \\ \hline
%                     \textit{regex}       & Regexes of varying complexity. \\ \hline
%                     \textit{random-text} & Random, likely syntactically \textit{in}valid text. \\ \hline
%                     \textit{random-ast}  & Random, but semantically \textit{valid} text. \\ \hline
%                 \end{tabular}
%                 \caption{\generator{} generators}
%                 \label{tbl:generators}
%             \end{table}

%         \item[\transformer{}] \hfill \\
%             This tool transforms \smt{} problems. It supports several transformers and options that specify its output and input. Table~\ref{tbl:transformers} describes them. The following transformers also preserve problem satisfiability: \textit{translate}, \textit{reverse}. \cite{ifaz}
%             \hfill \\
% 
%             \begin{table}
%                 \centering
%                 \begin{tabular}{|l|l|}
%                     \hline
%                     \textbf{Name}        & \textbf{The transformer ...} \\ \hline
%                     \textit{fuzz}        & Randomly replaces literals and operators with similar ones. \\ \hline
%                     \textit{graft}       & For each sort, swaps a random non-leaf node with a leaf node. \\ \hline
%                     \textit{multiply}    & Multiplies each number by N, and repeats each character N times. \\ \hline
%                     \textit{nop}         & Does nothing. \\ \hline
%                     \textit{reverse}     & Reverses all string literals and concat arguments. \\ \hline
%                     \textit{rotate}      & Performs a random syntax tree rotation on two concats. \\ \hline
%                     \textit{translate}   & Maps every character X to a different character Y. \\ \hline
%                     \textit{unprintable} & Randomly replaces all characters in literals with unprintable ones. \\ \hline
%                 \end{tabular}
%                 \caption{\transformer{} transformers}
%                 \label{tbl:transformers}
%             \end{table}

%         \item[\texttt{stringstats}] \hfill \\
%             This tool takes as input an \smt{} problem, and outputs its properties, such as: numbers of string variables and literals, max/median syntactic depth of expressions, and max/median length of literals.
%             \hfill \\

        % \item[\texttt{stringmerge}] \hfill \\
        %     This tool takes as input two \smt{} problems and combines them into one problem.

        %     \todo{Federico: describe \texttt{stringmerge}.}

%     \end{description}
