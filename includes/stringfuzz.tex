\section{\fuzzer{}}

    This section describes the fuzzer \fuzzer{}, which generates new \smt{} problems and transforms existing ones. \fuzzer{} is implemented as a Python package, and comes with several executables to generate, transform, and analyse \smt{} string problems.

    % Its source code repository is available at \sourceRepo{}. It can either be installed from source, or from the Python PIP package repository.

    The \fuzzer{} code is organised to be easily extended: all generators and transformers are independent modules. It is also implemented as a Unix ``filter'', to enable easy integration with other tools (including itself). For example, the outputs of generators can be piped directly into transformers, and transformers can be chained to produce a tuned stream of challenging inputs to a solver.

    Below are descriptions of the tools that constitute \fuzzer{}:

    \begin{description}

        \item[\texttt{stringfuzzg}] \hfill \\
            This tool generates \smt{} problems. It supports several generators and options that control its output. The generators it implements are: \\

            \todo{Fill in descriptions.}

            \begin{tabular}{|c|c|}
                \hline
                \textbf{Name} & \textbf{Description} \\ \hline
                concats       & \\ \hline
                lengths       & \\ \hline
                overlaps      & \\ \hline
                regex         & \\ \hline
                random-text   & \\ \hline
            \end{tabular}

            \hfill \\

        \item[\texttt{stringfuzzx}] \hfill \\
            This tool transforms \smt{} problems.. It supports several transformers and options that control its output and input. The transformers it implements are: \\

            \todo{Fill in descriptions.}

            \begin{tabular}{|c|c|}
                \hline
                \textbf{Name} & \textbf{Description} \\ \hline
                fuzz          & \\ \hline
                graft         & \\ \hline
                multiply      & \\ \hline
                nop           & \\ \hline
                reverse       & \\ \hline
                rotate        & \\ \hline
                translate     & \\ \hline
                unprintable   & \\ \hline
            \end{tabular}

            \hfill \\

        \item[\texttt{stringstats}] \hfill \\
            This tool takes as input an \smt{} problem, and outputs its properties. The properties are:

            \begin{itemize}
                \item number of string variables
                \item number of string literals
                \item max/median syntactic depth of expressions
                \item max/median length of literals
            \end{itemize}

            \hfill

        \item[\texttt{stringmerge}] \hfill \\
            This tool takes as input two \smt{} problems and combines them into one problem.

            \todo{Add details about merging strategies.}

    \end{description}
