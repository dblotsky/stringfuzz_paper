\section{\fuzzer{}}
\label{sec:fuzzer}

\subsubsection{Implementation and Architecture}

\fuzzer{} is implemented as a Python package, and comes with several
executables to generate, transform, and analyze \smtfull{} string and regex
instances. Its components are implemented as \unix{} ``filters'' to enable easy
integration with other tools (including themselves). For example, the
outputs of generators can be piped into transformers, and transformers
can be chained to produce a stream of tuned inputs to a
solver. \fuzzer{} is composed of the following tools:
\begin{description}
    \item[\generator{}] \hfill \\
    This tool generates \smt{} instances. It supports several generators and
    options that specify its output. Details can be found in
    Table~\ref{tbl:generators}.
    \item[\transformer{}] \hfill \\
    This tool transforms \smt{}
    instances. It supports several transformers and options that specify
    its output and input, which are explained in
    Table~\ref{tbl:transformers}. Note that transformers
    \textit{Translate} and \textit{Reverse} also preserve
    satisfiability under certain conditions.
    \item[\texttt{stringstats}] \hfill \\
    This tool takes an \smt{}
    instance as input and outputs its properties: the number of
    variables/literals, the max/median syntactic depth of expressions, the
    max/median literal length, etc.
\end{description}
We organized \fuzzer{} to be easily extended. As evidence of our success we note
that while the whole project
contains \linesInFuzzer{} lines of code, it takes an average of
\linesPerX{} lines of code to create a transformer. \fuzzer{} can either be
installed from source, or from the Python PIP package repository.

\begin{table}[t]
    \caption{\fuzzer{} built-in (a) generators and (b) transformers.}
    \begin{subtable}{1\textwidth}
        \centering
        \caption{\generator{} built-in generators.}
        \label{tbl:generators}
        \begin{tabular}{ l l }
            \toprule
            \textbf{Name}
            & \textbf{Generates instances that have ...} \\
            \midrule
            \textit{Concats}
            & Long concats and optional random extracts. \\
            \textit{Lengths}
            & Many variables (and their concats) with length constraints. \\
            \textit{Overlaps}
            & An expression of the form A.X = X.B. \\
            \textit{Equality}
            & An equality among concats, each with variables or constants. \\
            \textit{Regex}
            & Regexes of varying complexity. \\
            \textit{Random-Text}
            & Random, likely syntactically \textit{in}valid text. \\
            \textit{Random-AST}
            & Random, but semantically \textit{valid} text. \\
            \bottomrule
        \end{tabular}
    \end{subtable}
    \begin{subtable}{1\textwidth}
        \centering
        \caption{\transformer{} built-in transformers.}
        \label{tbl:transformers}
        \begin{tabular}{l l}
            \toprule
            \textbf{Name}
            & \textbf{The transformer ...} \\
            \midrule
            \textit{Fuzz}
            & Replaces literals and operators with similar ones.\\
            \textit{Graft}
            & Randomly swaps non-leaf nodes with leaf nodes.\\
            \textit{Multiply}\footnote{Can guarantee satisfiable output
            instances from satisfiable input instances \cite{website}.}
            & Multiplies integers and repeats strings by N.\\
            \textit{Nop}
            & Does nothing (can translate between \smtfull{}).\\
            \textit{Reverse}\footnote{Can guarantee input and output
            instances will be equisatisfiable \cite{website}.}
            & Reverses all string literals and concat arguments.\\
            \textit{Rotate}
            & Rotates compatible nodes in syntax tree.\\
            \textit{Translate}\footnotemark[6]
            & Permutes the alphabet.\\
            \textit{Unprintable}
            & Replaces characters in literals with unprintable ones.\\
            \bottomrule
        \end{tabular}
    \end{subtable}
\end{table}

\subsubsection{Regex Generating Capabilities}

\fuzzer{} can generate and transform instances with regex
constraints. For example, the command
``\texttt{stringfuzzg regex -r 2 -d 1 -t 1 -M 3 -X 10}'' produces this instance:
{\small\begin{verbatim}        (set-logic QF_S)
        (declare-fun var0 () String)
        (assert (str.in.re var0 (re.+ (str.to.re "R5"))))
        (assert (str.in.re var0 (re.+ (str.to.re "!PC"))))
        (assert (<= 3 (str.len var0)))
        (assert (<= (str.len var0) 10))
        (check-sat)\end{verbatim}}

Each instance is a set of one or more regex constraints on a single
variable, with optional maximum and minimum length constraints. Each regex
constraint is a concatenation (\texttt{re.++} in SMT-LIB string syntax)
of regex terms:
\begin{align*}
    & \texttt{(re.++ T1 (re.++ T2} \; ... \; \texttt{(re.++ Tn-1 Tn )))}
\end{align*}

\noindent and each term is recursively defined as any one of: repetition
(\texttt{re.*}), Kleene star (\texttt{re.+}), union (\texttt{re.union}), or a
character literal. Nested operators are nested up to a specified depth of
recursion. Terms at depth 0 are always regex constants.

Intuitively, each instance is a concatenation of regex terms, where each
term is a random nested regex operator with random arguments
(up to a specified depth) terminating in a regex literal.
Below are three example regexes (in regex syntax, not in the SMT-LIB language)
of depth 2 that can be produced by this scheme:
\begin{align*}
    & ((\texttt{a}|\texttt{b})|(\texttt{cc})+)\quad\quad\quad
    ((\texttt{ddd})*)+\quad\quad\quad ((\texttt{ee})+|(\texttt{fff})*)
\end{align*}

\subsubsection{Equisatisfiable String Transformations}
\fuzzer{} can also transform problem instances.
This is done by manipulating parsed syntax trees.
By default most of the built-in transformers
only guarantee well-formedness, however,
some can even guarantee equisatisfiability. Table~\ref{tbl:transformers}
lists the built-in transformers and notes these guarantees.

\subsubsection{Example Use Case}
In Sect.~\ref{sec:suites} we use \fuzzer{} to generate benchmark suites in a batch mode.
We can also use \fuzzer{} for on-line exploratory debugging.
For example, the script below repeatedly feeds random \fuzzer{}
instances to \cvc{} until the solver produces an error:
{\scriptsize\begin{verbatim}
while stringfuzzg -r random-ast -m \
    | tee instance.smt25 | cvc4 --lang smt2.5 --tlimit=5000 --strings-exp; do
    sleep 0
done\end{verbatim}}
