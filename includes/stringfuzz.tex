\section{\fuzzer{}}

    This section describes the fuzzer \fuzzer{}. \fuzzer{} is implemented as a Python package, and comes with several executables to generate, transform, and analyse \smt{} string problems. Its source code repository is available at \sourceRepo{}. It can either be installed from source, or from the Python PIP package repository.

    The \fuzzer{} code is organised to be easily extended with more generators and transformers. It is also implemented in the spirit of a Unix "filter", to be easily integrated into other tools. For example, the outputs of generators can be piped directly into transformers, and transformers can be chained to produce a tuned stream of challenging inputs to a solver.

    Below are descriptions of the tools that come with \fuzzer{}:

    \begin{description}

        \item[\texttt{stringfuzzg}] \hfill \\
            This tool generates \smt{} problems. It supports several generators and options that control its output. The generators it implements are:

            \begin{itemize}
                \item concats
                \item lengths
                \item overlaps
                \item regex
                \item random-text
            \end{itemize}

        \item[\texttt{stringfuzzx}] \hfill \\
            This tool transforms \smt{} problems.. It supports several transformers and options that control its output and input. The transformers it implements are:

            \begin{itemize}
                \item fuzz
                \item graft
                \item multiply
                \item nop
                \item reverse
                \item rotate
                \item translate
                \item unprintable
            \end{itemize}

        \item[\texttt{stringstats}] \hfill \\
            This tool takes as input an \smt{} problem, and outputs its properties. The properties are:

            \begin{itemize}
                \item number of string variables
                \item number of string literals
                \item max/median syntactic depth of expressions
                \item max/median length of literals
            \end{itemize}

            \hfill

        \item[\texttt{stringmerge}] \hfill \\
            This tool takes as input two \smt{} problems and combines them into one problem.

    \end{description}
