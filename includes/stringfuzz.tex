\section{\fuzzer{}}

    This section presents \fuzzer{}. \fuzzer{} is implemented as a Python package, and comes with several executables to generate, transform, and analyse \smt{} string problems. Its source code repository is available at \sourceRepo{}. It can either be installed from source, or from the Python PIP package repository.

    \subsection{Generators: \generator{}}

        \generator{} is available as a command-line tool, \texttt{stringfuzzg}. It supports several generators and options that control its output. The generators it implements are:

        \begin{itemize}
            \item concats
            \item lengths
            \item overlaps
            \item regex
            \item random-text
        \end{itemize}

    \subsection{Transformers: \transformer{}}

        \transformer is available as a command-line tool, \texttt{stringfuzzx}. It supports several transformers and options that control its output and input. The transformers it implements are:

        \begin{itemize}
            \item concats
            \item lengths
            \item overlaps
            \item regex
            \item random-text
        \end{itemize}

    \subsection{Utilities}

        Some utilities are also provided with \fuzzer{}. They are: \texttt{stringstats}, \texttt{stringmerge}, \texttt{smtparse}, and \texttt{smtscan}.

        \begin{description}

            \item[\texttt{stringstats}] \hfill \\ \\
                This tool takes as input an \smt{} problem, and outputs its properties. The properties are:

                \begin{itemize}
                    \item number of string variables
                    \item number of string literals
                    \item max/median syntactic depth of expressions
                    \item max/median length of literals
                \end{itemize}

                \hfill

            \item[\texttt{stringmerge}] \hfill \\ \\
                This tool takes as input two \smt{} problems and combines them into one problem.

        \end{description}

    \subsection{Architecture}

        Should I describe its architecture?
