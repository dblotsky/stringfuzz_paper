\documentclass{llncs}

\usepackage{amsmath}
\usepackage{url}
\usepackage{makeidx}
\usepackage{multirow}
\usepackage{graphicx}
\usepackage{float}
\usepackage{titling}
\usepackage{array}
\usepackage{filecontents}
\usepackage{varwidth}
\usepackage{enumitem}

\usepackage{subcaption}
\captionsetup{compatibility=false}

% define dimensions
\setlength{\pdfpageheight}{\paperheight}
\setlength{\pdfpagewidth}{\paperwidth}

% % configure minted
% \usepackage{minted}
% \setminted[bash]{linenos}

% constants
\def\fuzzer{StringFuzz}
\def\generator{\texttt{stringfuzzg}}
\def\transformer{\texttt{stringfuzzx}}

\def\smtfull{SMT-LIB 2.0/2.5}
\def\smt{SMT-LIB}
\def\unix{UNIX}

\DeclareMathOperator{\eval}{eval}
\def\br#1{\left(#1\right)} % Parenthesis

\def\theSuites{\textit{Concats-Balanced}, \textit{Concats-Small},
  \textit{Concats-Extracts-Small}, and \textit{Different-Prefix}}
\def\numSuites{four}

\def\cvc{CVC4}
\def\us{Z3str3}
\def\usOld{Z3str2}
\def\norn{Norn}
\def\theSolvers{\us{}, \cvc{}, \usOld{}, and \norn{}}
\def\numSolvers{four}

\def\linesPerX{45}
\def\linesInFuzzer{3,183}

% \def\problemRepo{\url{https://example.com}}
% \def\sourceRepo{\url{https://github.com/dblotsky/stringfuzz}}

\begin{document}

    % styles
    \pagestyle{headings} % switches on printing of running heads
    \addtocmark{Hamiltonian Mechanics} % additional mark in the TOC

    % title
    \title{
        \fuzzer{}: A Fuzzer for String Solvers
    }
    \titlerunning{\fuzzer{}} % abbreviated title (for running head)

%    % authors
%    \author{
%        Dmitry Blotsky\inst{1}\orcidID{0},
%        Federico Mora\inst{2}\orcidID{0},
%        Murphy Berzish\inst{1}\orcidID{0},\and
%        Yunhui Zheng\inst{3}\orcidID{0},
%        Ifaz Kabir\inst{1}\orcidID{0},
%        Vijay~Ganesh\inst{1}\orcidID{0}
%    }
%    \authorrunning{Dmitry Blotsky et al.}

%    % institutions
%    \institute{
%        University of Waterloo, Waterloo ON, Canada,\\
%        \email{dblotsky@uwaterloo.ca}
%        % U of T for Federico
%        % IBM for Yunhui
%    }

    \maketitle

    \begin{abstract}

    Many SMT solvers that support the theory of strings have been developed over the past decade, and several have matured to the point of being performant and practically useful. Moreover a standard input language, SMT-LIB, has been developed, and many mature solvers support it. In this paper we present StringFuzz: a fuzzer and SMT-LIB problem generator for string solvers. With it we analyse four mature string solvers: Z3str3, Norn, CVC4, and ABC. We use it to craft inputs that elicit poor performance in Z3str3 and CVC4, and provide algorithmic analyses of the causes of the performance degradations. Finally, we present several minor bugs in Z3str3 that were exposed by StringFuzz, to justify its further use by other solver authors to likewise improve their solvers.

\end{abstract}

    \section{Introduction}

    Many SMT solvers that support the theory of strings have been developed over the past decade, and several have matured to the point of being performant and practically useful. Moreover a standard input language, \smt{}, has been developed and many mature solvers support it.

    In this paper, we present the following contributions:

    \begin{enumerate}
        \item \fuzzer{}: a fuzzer of \smt{} string and regex problems.
        \item Generated inputs that elicit poor performance in \us{} and \cvc{}.
        \item A recommended fix to \us{} that alleviates the poor performance.
        \item A large suite of \smt{} string and regex problems we generated with \fuzzer{}.
    \end{enumerate}

    \section{\fuzzer{}}

    This section presents \fuzzer{}. \fuzzer{} is implemented as a Python package, and comes with several executables to generate, transform, and analyse \smt{} string problems. Its source code repository is available at \sourceRepo{}. It can either be installed from source, or from the Python PIP package repository.

    \subsection{Generators: \generator{}}

        \generator{} is available as a command-line tool, \texttt{stringfuzzg}. It supports several generators and options that control its output. The generators it implements are:

        \begin{itemize}
            \item concats
            \item lengths
            \item overlaps
            \item regex
            \item random-text
        \end{itemize}

    \subsection{Transformers: \transformer{}}

        \transformer is available as a command-line tool, \texttt{stringfuzzx}. It supports several transformers and options that control its output and input. The transformers it implements are:

        \begin{itemize}
            \item concats
            \item lengths
            \item overlaps
            \item regex
            \item random-text
        \end{itemize}

    \subsection{Utilities}

        Some utilities are also provided with \fuzzer{}. They are: \texttt{stringstats}, \texttt{stringmerge}, \texttt{smtparse}, and \texttt{smtscan}.

        \begin{description}

            \item[\texttt{stringstats}] \hfill \\ \\
                This tool takes as input an \smt{} problem, and outputs its properties. The properties are:

                \begin{itemize}
                    \item number of string variables
                    \item number of string literals
                    \item max/median syntactic depth of expressions
                    \item max/median length of literals
                \end{itemize}

                \hfill

            \item[\texttt{stringmerge}] \hfill \\ \\
                This tool takes as input two \smt{} problems and combines them into one problem.

        \end{description}

    \subsection{Architecture}

        Should I describe its architecture?

    \section{Instance Suites}
\label{sec:suites}

In this section, we describe the benchmark suites we generated
with \fuzzer{}, and on which we conducted our experimental
evaluation.\footnote{The link to the repository will be added after
double-blind review.}

Table~\ref{tbl:generated} describes the string and regex instances that were
generated by \generator{}. Table~\ref{tbl:transformed} lists benchmarks
derived from existing seed instances by iteratively applying \transformer{}.
Every transformed instance is named according to its
seed and the transformations it undertook. For example, input
\texttt{z3-regex-1-fuzz-graft.smt2} was transformed by applying
\textit{Fuzz} and then \textit{Graft} to \texttt{z3-regex-1.smt2}.

The \textit{Amazon} category contains 472 instances derived from two seeds
supplied by our industrial collaborators. The \textit{Regex} category is
seeded by the \usOld{} regular expression test suite~\cite{z3str2-tests}. This
test suite contains 42 instances dealing with a wide variety of regular
expressions. Through cumulative transformations we expanded the 42 seeds to 7,551 unique
instances. Finally, the \textit{Sanitizer} category is obtained from
five industrial e-mail address and IPv4 sanitizers.

\begin{table}[t]
    \caption{Repository of 15,843 \smtfull{} instances.}
    \begin{subtable}{1\textwidth}
        \centering
        \caption{\generator{}-generated instances.}
        \label{tbl:generated}
        \begin{tabular}{llr}
            \toprule
            \textbf{Name}
                & \textbf{Instances have a ...}
                & \textbf{Quantity}
            \\
            \midrule
            \textit{Concats}
                & Right-heavy, deep tree of concats.
                & 600 \\
            \textit{Concats-Balanced}
                & Balanced, deep tree of concats.
                & 100 \\
            \textit{Concats-Extracts}
                & Single concat tree, with character extractions.
                & 600 \\
            \textit{Lengths}
                & Single, large length constraint on a variable.
                & 1,000 \\
            \textit{Lengths-Concats}
                & Tree of fixed-length concats of variables.
                & 500 \\
            \textit{Overlaps}
                & Formula of the form A.X = X.B.
                & 100 \\
            \textit{Regex}
                & Complex regex membership test.
                & 1,200 \\
            \textit{Many-Regexes}
                & Multiple random regex membership tests.
                & 600 \\
            \textit{Regex-Deep}
                & Regex membership test with many nested operators.
                & 450 \\
            \textit{Regex-Pair}
                & Test for membership in one regex, but not another.
                & 600 \\
            \textit{Regex-Lengths}
                & Regex membership test, and a length constraint.
                & 600 \\
            \textit{Different-Prefix}
                & Equality of two deep concats with different prefixes.
                & 300 \\
            \bottomrule
        \end{tabular}
    \end{subtable}

    \begin{subtable}{1\textwidth}
        \centering
        \caption{\transformer{}-generated instances.}
        \label{tbl:transformed}
        \begin{tabular}{llr}
            \toprule
            \textbf{Name}      & \textbf{Seed}                              & \textbf{Quantity} \\
            \midrule
            \textit{Amazon}    & Two industrial regex membership instances. & 472 \\
            \textit{Regex}     & Z3str2 regular expression test suite.      & 7,551 \\
            \textit{Sanitizer} & Five e-mail and IPv4 sanitiser examples.   & 1,170 \\
            \bottomrule
        \end{tabular}
    \end{subtable}
\end{table}

    \section{Experimental Results and Analysis}
\label{sec:data}

We generated several problem suites with \fuzzer{} that made one
solver perform poorly, but not others. These suites are
\theSuites{}. Figure~\ref{fig:cvc-hard} shows the suites that were
uniquely difficult for \cvc{}. Figure~\ref{fig:z3str3-hard} shows the
suites that were uniquely difficult for \us{}. All experiments were
run in series, with a timeout of 15 seconds, on the same computer running
Ubuntu Linux 16.04. The computer had 32GB of RAM and an
Intel\textregistered{} Core\texttrademark{} i7-6700 CPU with clock speed
of 3.40GHz.

\begin{figure}[h]
    \begin{subfigure}{.5\textwidth}
        \includegraphics[width=\textwidth]{data/graphs/concats-extracts-small.png}
        \caption{Performance on concats-extracts-small}
        \label{fig:concats-extracts-small}
    \end{subfigure}
    \begin{subfigure}{.5\textwidth}
        \includegraphics[width=\textwidth]{data/graphs/different-prefix.png}
        \caption{Performance on different-prefix}
        \label{fig:different-prefix}
    \end{subfigure}
    \caption{Problems hard for \cvc{}}
    \label{fig:cvc-hard}

    \begin{subfigure}{.5\textwidth}
        \includegraphics[width=\textwidth]{data/graphs/concats-balanced.png}
        \label{fig:concats-balanced}
        \caption{Performance on concats-balanced}
    \end{subfigure}
    \begin{subfigure}{.5\textwidth}
        \includegraphics[width=\textwidth]{data/graphs/concats-small.png}
        \label{fig:concats-small}
        \caption{Performance on concats-small}
    \end{subfigure}
    \caption{Problems hard for \us{}}
    \label{fig:z3str3-hard}
\end{figure}

\subsubsection{Usefulness to \us{}: A Case Study}

We found a number of performance-related issues and opportunities for
new heuristics in \us{} thanks to \fuzzer{}. For example, the
instances in the \textit{concats-big} suite generated by \fuzzer{}
helped us discover a missing heuristic. In particular, \us{} didn't
make full use of the solving context (e.g. some terms are empty
strings) to simplify the concatenations of a long list of string terms
before trying to reason about the equivalences among subterms. \us{}
therefore introduced a large number of unnecessary intermediate
variables and propagations. As an immediate consequence of
testing \us{} on \textit{concats-big} benchmarks, we were able to
identify the factors common to these instances and rapidly formulate a
hypothesis as to why they were performing poorly.

    \section{Related Work}

    Naturally, many solver developers author their own test suites to validate their solvers \cite{cvc4-tests} \cite{z3str3-tests} \cite{z3str2-tests}. In addition, several popular problem suites are publicly available for solver validation, such as the Kaluza \cite{kaluza} and Kausler \cite{kausler} suites.

    There are likewise several fuzzers and problem generators currently available, but none of them can generate or transform string and regex problems. For example the FuzzSMT\cite{fuzzsmt} tool generates \smt{} problems with bit vectors and arrays, but does not support strings or regexes. The SMTpp\cite{smtpp} tool preprocesses and simplifies problems, but does not generate new ones or fuzz existing ones.

% FuzzSMT: http://fmv.jku.at/papers/BrummayerBiere-SMT09.pdf
% SMTpp: http://www.verit-solver.org/papers/smt2015.pdf


    \bibliographystyle{abbrv}
    \bibliography{paper}

    \section{Appendix}

In this section, we will prove some of the guarantees that transformations
provide in \fuzzer{}.

We assume that the reader is familiar with SMT-LIB syntax. The transformations
work over boolean, integer, string literals and expressions, and regular
expressions.

We will make the folowing simplifications for our proofs:
\begin{itemize}
  \item
    Since many of the operators can be expressed in terms of other, our
    proofs will only consider a subset of the operators. In particular we will
    drop the following:
    \begin{itemize}
      \item \texttt{|} (boolean or)
      \item \texttt{=} for boolean expressions
      \item \texttt{<=}, \texttt{>}, \texttt{>=}
      \item \texttt{str.prefix}, \texttt{str.suffix}, \texttt{str.at}
      \item \texttt{str.contains}
      \item \texttt{re.range}
    \end{itemize}

  \item
    In the above, we assume that \texttt{str.at} is expressible using
    \texttt{substring} although the semantics slightly different when the index
    is out of bounds.

  \item
    We will assume that our program instances are a single assertion. Multiple
    assertions can be expressed by combining them using \texttt{and}.
\end{itemize}

\begin{definition}
  A \emph{model} is a mapping from boolean, integer, and string variables to
literals of the corresponding type.
\end{definition}

\begin{definition}
  A \emph{program} $P$ satisfies a model $m$ if $P$ evaluates to \texttt{true}
under the model $m$. The function $\eval (P)$ evaluates a program instance
under a model; the specific model used is left implicit.
\end{definition}

The proofs all follow a very similar structure. For each transformation
\texttt{Trans}, we show that a problem instance $P$ is satisfiable by a model $m$
if and only if $\texttt{Trans}(P)$ is satisfiable by $\texttt{Trans}(m)$, where
$\texttt{Trans}(M)$ is the transformation applied to all the constants in the
model $m$. In other words, we show that $\eval (P) = \eval (\texttt{Trans}(P))$.
This is proven using a straightforward induction on the structure of problem
instances.

\texttt{Multiply} has a weaker guarantee; it takes satisfiable problems to
satisfiable problems. For instance, $y<x<y+1$ is UNSAT, but multiplying by 2
transforms the problem into $y<x<y+2$ which is SAT. For \texttt{Multiply}, we
prove that if $\eval (P)$ is \texttt{true}, then, $\eval \texttt{Multiply}(P)$
is also \texttt{true}.

\subsection{Translate}
We now prove that translate transforms programs into equisatisfiable programs.


    % \appendix
    % \section{Appendix}

\end{document}
